\section{Problem}
\subsection{Background}
\begin{frame}{Problem}
  \begin{itemize}
    \item Many practical computing problems concern large scale graphs
    \item Efficient processing of such large scale graphs is a challenge
    \item  There are different frameworks e.g. Pregel, Giraph, GraphX, Mizan, GraphLab, GPS, PowerGraph, etc. that have been developed for this purpose
  \end{itemize}
  \end{frame}

  \begin{frame}{Problem}
    \begin{itemize}
    \item There are graph based algorithms that can be performed in a fully vertex-centric and hence parallel fashion (e.g. shortest path, pagerank etc.)
    \item However, some graph algorithms are a combination of vertex-centric (parallel) and global (sequential) computations (e.g. k-clustering, affinity propagation etc.)
    \item Some of the frameworks dedicated for processing graphs also facilitate graph mutation operations
        \begin{itemize}
          \item e.g. Pregel, Mizan, Giraph, GPS, GraphLab
        \end{itemize}
      \end{itemize}
    \end{frame}

\subsection{Context}
    \begin{frame}{Problem}
    \begin{itemize}
    \item Topology mutation of graph introduces concurrent graph mutation conflicts
    \newline
     \item Multiple vertices may issue conflicting requests in the same superstep
     	\begin{itemize}
		\item e.g., two requests to add a vertex V , with different initial values
	\end{itemize}
 \end{itemize}
\end{frame}

    \begin{frame}{Problem}
	Current frameworks handle concurrent conflicts through two mechanisms:          			\begin{itemize}
	            	\item Partial Ordering:
		    		\begin{itemize}
					\item Removals are performed first, edge removal before vertex removal
					\item Addition follows the order, where vertex addition precedes edge addition
				\end{itemize}
            \item User defined Handlers:
            		\begin{itemize}
					\item Default is randomly allowing choosing one request out of conflicted ones
					\item Users with special needs may specify other conflict resolution policy
         		 \end{itemize}
		 \end{itemize}
\end{frame}
